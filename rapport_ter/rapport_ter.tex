\documentclass{article}
\usepackage[utf8]{inputenc}
\usepackage{fullpage}
\title{Simulation Numérique de l'amortissement Landau dans les plasmas}
\date{2019}
\author{Yassine MOUFTAH\\Guillaume ROUMAGE}
\begin{document}
\maketitle
\newpage
\noindent Dans tout le document, les problèmes seront étudiés sur un intervalle de temps noté $[0;T]$, avec $T \in ]0; + \infty[$. Pour discrétiser le problème afin de faire des simulations numériques, on choisit un nombre de pas $N$. L'intervalle $[0;T]$ est subdivisé en $N$ sous-intervalles. Le pas de temps, représenté par $h$, est alors donné par la formule : $h = \frac{T}{N}$.
\newpage
\section*{Introduction}
Il est courant de dire que la matière possède trois états : l'état solide, l'état liquide, ainsi que l'état gazeux. Mais en réalité, il existe un quatrième état de la matière : c'est le plasma.\\
Comme nous le savons, les atomes sont composés d'un noyau, contenant des protons et des neutrons, autour duquel gravitent des électrons. A l'état solide, liquide ou gazeux, les électrons qui orbitent autour des atomes restent sur leurs orbites. Lorsque de la matière est chauffé, elle passe successivement de l'état solide à l'état liquide, puis de l'état liquide à l'état gazeux. Lorsque le gaz est chauffé à très haute température (supérieure à $10^4$ K), il devient un plasma.\\
Cette haute température libèrent les électrons des l'orbites de l'atome auquel ils étaient rattaché, qui peuvent désormais se déplacer dans le gaz.\\
\\
L'analyse numérique permet de reproduire par le calcul une réalité physique. Cette simulation peut être effectuée pour plusieurs raisons : l'expérience peut-être irréalisable, avoir un coût trop élevé, ou encore pour prévoir grâce aux ordinateurs comment l'expérience va se dérouler, avant de faire cette même expérience en laboratoire, en conditions réelles.
\subsection*{Constante}
\begin{itemize}
\item Charge élémentaire d'un électron : $q_e = -1.6021766208.10^{-19} C$.
\item Masse d'un électron : $m_e = 9.109.10^{-31} kg$.
\end{itemize}
\section*{Vlasov}
\subsection*{Mouvement de particules en 1D}
Nous simulons en premier lieu le mouvement d'une, puis de plusieurs particules en une dimension, c'est-à-dire sur un segment défini. Si la particule atteint le bord droit du segment, elle est envoyée sur le bord gauche, et inversement. Cela permet à la particule reste dans l'intervale d'espace étudié. Les particules étudiées seront les électrons.\\
Les équations donnant la position et la vitesse des électrons sont données par les systèmes suivants. Tout d'abord, en faisant l'hypothèse que le champs électrique auquel est soumis l'électron est nul.
$$
\left \{
   \begin{array}{l l l}
      x'(t)  & = & v(t) \\
      v'(t)  & = & 0 \\
	\end{array}
\right.
$$
avec $t \in [0;T]$.\\
En discrétisant et en utilisant la méthode d'Euler explicite, cela se réécrit :
$$
\left \{
   \begin{array}{l l l}
      x_{n+1}  & = & x_n + h.v_n \\
      v_{n+1}  & = & v_n \\
	\end{array}
\right.
$$
avec $n \in \{0,...,N-1\}$.\\
Ajoutons maintenant un champ électrique constant. Les équations décrivant la position et la vitesse des électrons sont données par :
$$
\left \{
   \begin{array}{l l l}
      x'(t)  & = & v(t) \\
      v'(t)  & = & -\frac{q_e}{m_e}.E_0(x(t)) \\
	\end{array}
\right.
$$
avec $t \in [0;T]$, $m_e$ la masse d'un électron, $q_e$ la valeur de la charge élémentaire de l'électron, $E_0$ la fonction représentant la valeur du champ électrique en fonction de la position.\\
Dans la simulation, la fonction $E_0$ utilisée est $E_0 : t \mapsto (-10^{-3}.t.(t-L) + 0.1) \times \frac{-m_e}{q_e}$.\\
Soit en discrétisant et en utilisant la méthode d'Euler explicite :
$$
\left \{
   \begin{array}{l l l}
      x_{n+1}  & = & x_n + h.v_n \\
      v_{n+1}  & = & -\frac{q_e}{m_e}.E_0(x_n) \\
	\end{array}
\right.
$$
\end{document}
